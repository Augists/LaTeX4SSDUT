% !TEX TS-program = XeLaTeX
% !TEX encoding = UTF-8 Unicode


\cdegree{\makebox[13.25cm][s]{大连理工大学本科毕业设计(论文)}}
\ctitle{基于Wi-Fi和视觉的多模态行为识别方法研究}
\etitle{Research on Multi-modal Human Activity Recognition Method Based on Wi-Fi and Vision}

% 根据需要添加字符间距
\department{\makebox[6.1cm][c]{软件学院}}   
\csubject{\makebox[6.1cm][c]{网络工程}}
\cauthor{\makebox[6.1cm][c]{Augists}}
\cauthorno{\makebox[6.1cm][c]{201992222}}
\csupervisor{\makebox[6.1cm][c]{大老板}}
\teacher{\makebox[6.1cm][c]{小老板}}
% \cdate{\makebox[6.1cm][c]{2023年5月28日}}
\cdate{{\quad\quad\;}\the\year~年~\the\month~月~\the\day~日}  % 中文日期
% \cdate{\makebox[6.1cm][c]{\today}}  % 英文日期

\cabstract{
“摘要”是摘要部分的标题,不可省略。

标题“摘要”选用模板中的样式所定义的“标题1”,再居中;或者手动设置成字体:黑体,居中,字号:小三,1.5倍行距,段后11磅,段前为0。

摘要是毕业设计(论文)的缩影,文字要简练、明确。内容要包括目的、方法、结果和结论。单位采用国际标准计量单位制,除特别情况外,数字一律用阿拉伯数码。文中不允许出现插图。重要的表格可以写入。

摘要正文选用模板中的样式所定义的“正文”,每段落首行缩进2个汉字;或者手动设置成每段落首行缩进2个汉字,字体:宋体,字号:小四,行距:多倍行距 1.25,间距:段前、段后均为0行,取消网格对齐选项。

摘要篇幅以一页为限,字数限500字以内。

摘要正文后,列出3-5个关键词。“关键词:”是关键词部分的引导,不可省略。关键词请尽量用《汉语主题词表》等词表提供的规范词。

关键词与摘要之间空一行。关键词词间用分号间隔,末尾不加标点,3-5个;黑体,小四,加粗。关键词整体字数限制在一行。
}

\ckeywords{多模态;人体行为感知;深度学习}

\eabstract{
\sloppy{}

% 随机英文段落
\lipsum[1-3]
}

\ekeywords{Key; Word; List}
\makecover 