% !TEX TS-program = XeLaTeX
% !TEX encoding = UTF-8 Unicode

% 英文字体设置特别推荐方案(Windows,需要安装 Adobe 字体),现代
% \usepackage[UTF8]{ctex}
\usepackage{fontspec}
\usepackage{xltxtra,xunicode}
\usepackage[CJKnumber,CJKchecksingle,BoldFont]{xeCJK}
\usepackage{amsmath}
\usepackage{amssymb}
\setmainfont[Mapping=tex-text]{Times New Roman}
\setsansfont[Mapping=tex-text]{Arial}
% \newfontfamily\tempus{FandolSong}
\newfontfamily\tempus{SIMSUN.TTC}  % 如果这里编译错误,请自行替换为系统中的宋体字体名
% \newfontfamily\ms{MS Sans Serif}
%\setmonofont{Consolas}


% 英文字体设置方案一(Windows,需要安装 LM10 字体),和 LaTeX 默认字体保持一致,经典
% \usepackage{amssymb}
% \usepackage{fontspec}
% \usepackage{amsmath}
% \usepackage[CJKnumber,CJKaddspaces,CJKchecksingle,BoldFont]{xeCJK}
% \usepackage{mathrsfs}   % 一种常用于定义泛函算子的花体字母,只有大写。
% \usepackage{bm}         % 处理数学公式中的黑斜体的宏包
% \setmainfont{LMRoman10-Regular}
% \setsansfont{LMSans10-Regular}
% \setmonofont{LMMono10-Regular}

% 英文字体设置方案二(Linux,使用自带 LM10 字体),和 LaTeX 默认字体保持一致,经典
% \usepackage{fontspec}
% \usepackage{amsmath,amssymb}
% \usepackage[CJKnumber,CJKaddspaces,CJKchecksingle,BoldFont]{xeCJK}
% \usepackage{mathrsfs}   % 一种常用于定义泛函算子的花体字母,只有大写。
% \usepackage{bm}         % 处理数学公式中的黑斜体的宏包
% \setmainfont{LMRoman10}
% \setsansfont{LMSans10}
% \setmonofont{LMMono10}

% 英文字体设置方案三(Linux,使用自带 Nimbus 字体),和 Word 模版字体保持一致,经典
% \usepackage{fontspec}
% \usepackage{mathptmx}
% \usepackage{amsmath,amssymb}
% \usepackage[CJKnumber,CJKaddspaces,CJKchecksingle,BoldFont]{xeCJK}
% \usepackage{mathrsfs}   % 一种常用于定义泛函算子的花体字母,只有大写。
% \usepackage{bm}         % 处理数学公式中的黑斜体的宏包
% \setmainfont{Nimbus Roman No9 L}
% \setsansfont{Nimbus Sans L}
% \setmonofont{Nimbus Mono L}

% 中文字体设置,使用的是 Adobe 字体,保证了在 Adobe Reader / Acrobat 下优秀的显示效果
\setCJKmainfont[BoldFont={SIMSUN.TTC},ItalicFont={SIMSUN.TTC}]{SIMSUN.TTC}
\setCJKsansfont{SIMHEI.TTF}
\setCJKmonofont{SIMSUN.TTC}
% \setCJKmainfont[BoldFont={FandolSong},ItalicFont={FandolSong}]{FandolSong}
% \setCJKsansfont{FandolHei}
% \setCJKmonofont{FandolSong}

% 定义字体名称,可在此添加自定义的字体
\setCJKfamilyfont{song}{SIMSUN.TTC}
\setCJKfamilyfont{hei}{SIMHEI.TTF}
\setCJKfamilyfont{hwxhei}{STXIHEI.TTF}
\setCJKfamilyfont{hwxkai}{STXINGKA.TTF}
% \setCJKfamilyfont{song}{FandolSong}
% \setCJKfamilyfont{hei}{FandolHei}
% \setCJKfamilyfont{hwxhei}{FandolHei}
% \setCJKfamilyfont{hwxkai}{FandolKai}
% \setCJKfamilyfont{FandolHei}{STXihei} 
% \setCJKfamilyfont{FandolKai}{华文行楷}
% \setCJKfamilyfont{FandolHei}[AutoFakeBold = {2.17}]{华文细黑}
%\setCJKfamilyfont{fs}{Adobe Fangsong Std}
%\setCJKfamilyfont{xkai}{STXingkaiSC-Bold}

% 自动调整中英文之间的空白
% \punctstyle{quanjiao}
\XeTeXlinebreaklocale "zh"      %中文断行
\XeTeXlinebreakskip = 0pt plus 1pt %1pt左右弹性间距

% 其他字体宏包
