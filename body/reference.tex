\song\wuhao
\setmainfont{SIMSUN.TTC}
\linespread{1.25}
% \setlist[enumerate]{[1],itemsep=-3pt,topsep=0mm,labelindent=\parindent,leftmargin=*}
% \chapter*{\hfill 参~考~文~献 \hfill}
\addcontentsline{toc}{chapter}{参~考~文~献}
\label{reference}
\sloppy{}

参考文献推荐使用bibtex或bibitem生成,如有需要可自行补充填写bibtex。另需注意,如果引用了硕士/博士毕业论文,需要添加城市信息,可参考reference.bib

% \begin{thebibliography}{99}
% \bibitem{khurana2018deep}Khurana R, Kushwaha A K S. Deep learning approaches for human activity recognition in video surveillance-a survey[C]//2018 First International Conference on Secure Cyber Computing and Communication (ICSCCC). IEEE, 2018: 542-544.
% \bibitem{baccouche2011sequential}Baccouche M, Mamalet F, Wolf C, et al. Sequential deep learning for human action recognition[C]//Human Behavior Understanding: Second International Workshop, HBU 2011, Amsterdam, The Netherlands, November 16, 2011. Proceedings 2. Springer Berlin Heidelberg, 2011: 29-39.
% \end{thebibliography}
% \vspace{17.06pt}

% 标题“参考文献”不可省略,选用模板中的样式所定义的“参考文献”;或者手动设置成字体:黑体,居中,字号:小三,1.5倍行距,段后1行,段前为0行。

% 参考文献内容设置成字体:宋体,字号:五号,多倍行距1.25,段前、段后均为0行,取消网格对齐选项。

% 参考文献的著录,按论文中引用顺序排列。

% 参考文献数量不少于10篇,其中期刊不少于5篇,并且包含一定数量的外文期刊。

% 文献类型标志参考国家标准 GB/T 7714-2005,如下表:

\vspace{17.06pt}
\bibliography{reference}


\setmainfont[Mapping=tex-text]{Times New Roman}